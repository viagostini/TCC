%!TeX root=../tese.tex
%("dica" para o editor de texto: este arquivo é parte de um documento maior)
% para saber mais: https://tex.stackexchange.com/q/78101/183146

% O resumo é obrigatório, em português e inglês. Este comando também gera
% automaticamente a referência para o próprio documento, conforme as normas
% sugeridas da USP
\begin{resumo}{port}
O problema do Ancestral de Nível consiste em preprocessar uma árvore enraizada
$T$ para então responder consultas AN$(u,d)$, que pedem o ancestral do nó $u$ com
profundidade $d$. Vários algoritmos com diferentes complexidades já foram propostos,
porém existem poucos estudos comparativos e implementações disponíveis, principalmente
em português. Neste trabalho, foram implementados e comparados quatro algoritmos,
levando em consideração árvores com diferentes formatos para analisar na prática o
comportamento de cada um conforme o fator de ramificação varia, indo além
da análise tradicional de complexidade de tempo e espaço no pior caso. Os resultados
obtidos estão de acordo com o que era esperado pela análise teórica mas além disso,
foi possível observar que algoritmos simples podem ser implementados de maneiras que,
dependendo das características da árvore de entrada, pode-se obter performances
suficientes, o que leva à reforçar o quão importante é conhecer a aplicação em
questão para poder tomar a melhor decisão quanto a qual algoritmo implementar,
balanceando questões como uso de memória, dificuldade de implementação e manutenção
do código e eficiência.

\end{resumo}

% O resumo é obrigatório, em português e inglês. Este comando também gera
% automaticamente a referência para o próprio documento, conforme as normas
% sugeridas da USP
\begin{resumo}{eng}
The Level Ancestor problem consists of preprocessing a rooted tree $T$ to answer
queries AN$(u,d)$ which requests the ancestor of node $u$ with depth $d$. There
have been several algorithms with different complexities proposed, however, there are
few comparative studies and implementations available, especially in portuguese. In
this study, four algorithms were implemented and tested with trees of varied shape
to analyse, in practice, each algorithm's behaviour as the branching factor changes,
going beyond the traditional worst-case time and space complexity analysis. The
obtained results were in line with what was expected from theoretical analysis, but
it allowed to observe that simple algorithms can be implemented in such ways that,
depending on the characteristics of the input tree, it can achieve sufficient
performance, reinforcing how important it is to know the actual application so that
an informed decision is made about which algorithm to implement, balancing questions
such as memory use, difficulty of code implementation and maintainability and it's
efficiency.
\end{resumo}
