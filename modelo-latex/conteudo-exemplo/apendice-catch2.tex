%!TeX root=../tese.tex
%("dica" para o editor de texto: este arquivo é parte de um documento maior)
% para saber mais: https://tex.stackexchange.com/q/78101/183146

\chapter{Testes de unidade com a biblioteca Catch2}
\label{ap:pseudocode}

Para garantir a corretude dos algoritmos implementados neste trabalho, já estava
decidido desde o início a utilizar alguma ferramenta para realizar testes que pudessem
facilmente identificar erros nas implementações ao longo do ano. Depois de pesquisar
bastante optei por utilizar a biblioteca \textbf{Catch2} por ser \textit{header-only}
e facilitar o processo de rodar em outras máquinas sem precisar instalar nada.
A sintaxe do \textit{framework} fornecido pela biblioteca é bem organizada e foi
de fácil utilização. No programa ~\ref{prog:tabletest} consta uma parte do código
real dos testes do Algoritmo da Tabela, exemplificando quão simples é escrever testes
de unidade bem separados por classe e tipo de teste, podendo criar diversos casos de
teste, cada um com um \textit{setup} diferente de variáveis e objetos.

Para este estudo, cada algoritmo teve seus testes encapsulados dentro de casos de teste
(\texttt{TEST\_CASE}) enquanto cada tipo de teste dentro de cada algoritmo (árvores lineares, binárias
e quaternárias) foram colocados dentro de suas próprias seções (\texttt{SECTION}). As asserções, que
fazem o papel de garantir que algo realmente vale é feito através das macros
\texttt{REQUIRE}, que checa igualdade entre dois valores e \texttt{REQUIRE\_THROW\_AS},
que verifica se uma função levantou a exceção que era esperada.

\begin{program}[H]
      \lstinputlisting[
        language=c++,
        style=pseudocode,
        style=wider,
        functions={TableAlgorithm, build_balanced_kary_tree, Tree},
        specialidentifiers={TEST_CASE, SECTION, REQUIRE, REQUIRE_THROWS_AS},
      ]
      {conteudo-exemplo/catch2-1.cpp}
    
      \caption{Parte dos testes para o Algoritmo da Tabela.\label{prog:tabletest}}
    \end{program}
